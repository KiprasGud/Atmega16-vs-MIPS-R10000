\documentclass{article}
\usepackage{graphicx} % Required for inserting images

\title{Atmega16 vs MIPS R10000}
\author{Kipras Gudaitis}
\date{2025}

\begin{document}

\maketitle

\section*{Įvadas}
Šiame darbe palyginsiu Atmega16 ir MIPS R10000 kompiuterių architektūras
\section{Elementinė kompiuterio/procesoriaus bazė, fizinės savybės}
\subsubsection*{Atmega16}
Atmega16 yra mažos galios CMOS 8-bitų mikrovaldiklis. Mikrovaldiklis turi 16 KB sistemos viduje programuojamos Flash atminties, 512-bitų EEPROM, 1kb SRAM. 10 tūkstančių Flash rašymo ištrynimo ciklų, 100 tūktančių EEPROM, gali data išsaugoti 20 metų 80C laipsnių temperatūroje ir 100 metų 25C laipsnių temperatūroje. Mikrovaldiklio svoris yra 1.6g, o energijos suvartojimas 1.1 mA, o miego rėžime 0.35mA, o darbinė įtampa nuo 4.5V iki 5.5V.
\subsubsection*{MIPS R10000}
R10000 yra 64-bitų mikroprocesorius. Jis turi maždaug 6.8 milijonus tranzistorių iš kurių apie 4.4 milijonai yra pirminėse talpyklose. Tai pat 2.8 milijonai tranzistorių yra logikos tanzistoriai. Lusto dydis yra 16.640 x 17.934 mm, o jo plotas 298.44 mm$^2$. Lustas sunaudoja 30w elektros esant 3.3V įtampai 200MHz dažniui.
\subsubsection*{Dalies apibendrinimas}
Matome, kad Atmega16 yra mažai galios suvartojantis parastas mikrovaldiklis, o MIPS R1000 galingas mikroprocesorius.

\section{Architektūros tipas}
\subsubsection*{Atmega16}
Atmega16 yra Harvardo stiliaus 8-bitų RISC Architektūra. Mikrovaldiklis naudoja registrus. Jis turi 32x8 bendros paskirties registrus. Jo visi 32 registrai yra sujungti su aritmetinės logikos įtaisu (Arithmetic Logic unit), tai jam leidžia gauti prieiga prie dviejų skirtingų registrų vienos instrukcijos vykdymo metu.
\subsubsection*{MIPS R10000}
R1000 mikroprocesorius naudoja MIPS ANDES architektūrą.Tai pat naudoja 64-bitų MIPS IV instrukcijų architektūrą. Jis per viena "clock-cycle" gali dekoduoti iki 4 instrukcijų. Mikroprocesorius naudoja dinaminį instrukcijų planavimą ir "Out-of-order execution".
\subsubsection*{Dalies apibendrinimas}
Matome, kad abiejų architektūrų pagrindas yra RISC architektūra, tačiau R10000 yra žymiai sudėtingesnė architektūra, kurios tikslas yra skaičiavimų kiekis, o Atmega16 tikslas yra būti patikimam ir paprastam.
\section{Adresai}
\subsubsection*{Atmega16}
Atmega16 dviejų adresų mašina.
\subsubsection*{MIPS R10000}
R10000 yra trijų adresų mašina.
\section{Registai}
\subsubsection*{Atmega16}
Atmega16 turi 32x8 bendros paskirties registrus, dar turi du 8-bitų "Timer/counters" registrus ir vieną 16-bitų "Timer-Counter registrą, . Tai pat atmega16 sujungia viršutinius 6 registrus, kad gautų registrų poras X,Y,Z, kurie vaidintų kaip 16 bitų adresai.
\subsubsection*{MIPS R10000}
R10000 turi 32x64 bendros paskirties sveikųjų skaičių registrus, ir 32x64 slankiojo kablelio skaičių registrus. Tai pat turi HI/LO registrus skirtus daugybai ir dalybai, 64-bitų PC registrą. 
\section{Požymių bitai}
\subsubsection*{Atmega16}
Atmega16 naudoja 8 požymių bitus, I - global interrupt, T - bit copy storage, S - sign, H - half carry, N - negative, Z - zero, C - carry
\subsubsection*{MIPS R10000}
Nenaudoja požymių bitų
\section{Architektūros duomenų plotis}
\subsubsection*{Atmega16}
Atmega16 turi 16-bitų žodį
\subsubsection*{MIPS R10000}
R10000 turi 32-bitų žodį
\section{Atminties išdėstymas}
\subsubsection*{}

\section{Virtuoli atmintis}
\subsubsection*{Atmega16}
Atmega16 neturi virtualios atminties.
\subsubsection*{MIPS R10000}
R10000 turi puslapinę virtualia atmintį. Virtuali atmintis yra 44-bitų, kas leidžia adresuoti iki 16TB virtualios atminties. Tai pat virtualios atminties adresai yra konvertuojami į 40-bitų fizinius adresus.

\section{Komandos}
\subsubsection*{Atmega16}
Atmega16 turi 8-bitų RISC komandų sistemą. Mikrovaldiklis turi šias klases: Data transfer, Arithmetic/Logic, Bit, Control, Rotate/Shift, Multiply, MCU. O jo keletas instrukcijų yra tokios: ADD Rd, Rr (Add without Carry); INC Rd (Increment),  ICALL (Indirect Call to (Z)); JMP k (Jump), MOV Rd, Rr (Copy Register); LSL Rd (Logical Shift Left); SEI (Global Interrupt Enable), SEV (Set Two’s Complement Overflow); SLEEP (Sleep), EOR Rd, Rr (Exclusive OR).
\subsubsection*{MIPS R10000}
R10000 naudoja MIPS IV architektūros komandų sistemą. Mikroprocesorius turi šias klases: Integer arithmetic/logic, Immediate, Load/Store,Branch/Jump, Coprocessor. Jo keletatas instrukcijų yra tokios: MOVZ (Move Conditional on Zero); TGEI (Trap if Greater Than or Equal Immediate); SYSCALL (System Call); BLTZ (Branch on Less Than Zero); J (Jump), JR (Jump Register); MULT (Multiply Word), DSLL32 (Doubleword Shift Left Logical + 32), ADD (Add Word), LL (Load Linked Word)
\subsubsection*{Dalies apibendrinimas}
Abi architektūros palaiko aritmetic/logical, brach, load/store operacijas. Tačiau Stmega16 turi bitų manipuliavimo operacijas, kurių R10000 neturi. Tai pat R10000 turi didesnį registrų failą.

\section{Adresavimo būdai}
\subsubsection*{Atmega16}
Atmega16 palaiko penkis adresavimo būdus: Tiesioginį, netiesioginį, netiesioginį su poslinkiu ir netiesioginį su pre-decrement ir post-increment, betarpį.
\subsubsection*{MIPS R10000}
R10000 palaiko penkis adresavimo būdus: Registro tiesioginį su poslinkiu, PC santykinį, absoliutų šuolį, registro šuolį,  betarpį.
\subsection*{I/O galimybės}
\subsubsection*{Atmega16}
Atmega16 turi 32 I/O linijas. I/O buvo tvarkomas per tiesioginį adresavimą. Turi 8 kanalų 10-bitų ADC, 3 laikmačius, 4 PWM kanalus, USART, SPI, TWI komunikacijos sąsajas.
\subsubsection*{MIPS R10000}
R10000 neturi išorinių įrenginių. I/O operacijos atliekamos per išorines magistrales. Jis susisiekia prie išorinių įrenginių per 64-bitų "Avalanche" magistralę, ji perduoda duomnenis į atmintį arba į I/O įrenginius.
\subsubsection*{Dalies apibūdinimas}
Palyginant Atmega16 turi savo I/O, o R10000 turėjo naudoti magistrales, kad galėtų susisiekti su I/O įrenginiais.

\end{document}
