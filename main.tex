\documentclass{article}
\usepackage{graphicx} % Required for inserting images

\title{Atmega16 vs MIPS R10000}
\author{Kipras Gudaitis}
\date{2025}

\begin{document}

\maketitle

\section*{Įvadas}
Šiame darbe palyginsiu Atmega16 ir MIPS R10000 kompiuterių architektūras
\section{Elementinė kompiuterio/procesoriaus bazė, fizinės savybės}
\subsubsection*{Atmega16}
Atmega16 yra mažos galios CMOS 8-bitų mikrovaldiklis. Mikrovaldiklis turi 16 KB sistemos viduje programuojamos Flash atminties, 512-bitų EEPROM, 1kb SRAM. 10 tūkstančių Flash rašymo ištrynimo ciklų, 100 tūktančių EEPROM, gali data išsaugoti 20 metų 80C laipsnių temperatūroje ir 100 metų 25C laipsnių temperatūroje. Mikrovaldiklio svoris yra 1.6g, o energijos suvartojimas 1.1 mA, o miego rėžime 0.35mA, o darbinė įtampa nuo 4.5V iki 5.5V.
\subsubsection*{MIPS R10000}
R10000 yra 64-bitų mikroprocesorius. Jis turi maždaug 6.8 milijonus tranzistorių iš kurių apie 4.4 milijonai yra pirminėse talpyklose. Tai pat 2.8 milijonai tranzistorių yra logikos tanzistoriai. Lusto dydis yra 16.640 x 17.934 mm, o jo plotas 298.44 mm$^2$. Lustas sunaudoja 30w elektros esant 3.3V įtampai 200MHz dažniui.
\subsubsection*{Dalies apibendrinimas}
Matome, kad Atmega16 yra mažai galios suvartojantis parastas mikrovaldiklis, o MIPS R1000 galingas mikroprocesorius.

\section{Architektūros tipas}
\subsubsection*{Atmega16}
Atmega16 yra Harvardo stiliaus 8-bitų RISC Architektūra. Mikrovaldiklis naudoja registrus. Jis turi 32x8 bendros paskirties registrus. Jo visi 32 registrai yra sujungti su aritmetinės logikos įtaisu (Arithmetic Logic unit), tai jam leidžia gauti prieiga prie dviejų skirtingų registrų vienos instrukcijos vykdymo metu.
\subsubsection*{MIPS R10000}
R1000 mikroprocesorius naudoja MIPS ANDES architektūrą.Tai pat naudoja 64-bitų MIPS IV instrukcijų architektūrą. Jis per viena "clock-cycle" gali dekoduoti iki 4 instrukcijų. Mikroprocesorius naudoja dinaminį instrukcijų planavimą ir "Out-of-order execution".
\subsubsection*{Dalies apibendrinimas}
Matome, kad abiejų architektūrų pagrindas yra RISC architektūra, tačiau R10000 yra žymiai sudėtingesnė architektūra, kurios tikslas yra skaičiavimų kiekis, o Atmega16 tikslas yra būti patikimam ir paprastam.
\section{Adresai}
\subsubsection*{Atmega16}
Atmega16 dviejų adresų mašina.
\subsubsection*{MIPS R10000}
R10000 yra trijų adresų mašina.

\end{document}
